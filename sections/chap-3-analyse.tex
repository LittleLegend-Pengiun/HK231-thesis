\chapter{Phân tích yêu cầu}
\section{Phân tích nghiệp vụ}
\section{Phân tích bài toán và đề xuất giải pháp}
\subsection{Định nghĩa các yêu cầu mục tiêu của đồ án}
\noindent Đề tài xây dựng nhằm phân tích và giải quyết bài toán xoay quanh 2 từ khóa \textbf{scalability} và \textbf{availability}. Do đó, ta cần đi vào tìm hiểu định nghĩa của 2 từ khóa này:
% \\[0.5cm]

\subsubsection{Availability - Tính sẵn sàng}
\noindent Theo định nghĩa của Microsoft\footnote{Website: https://learn.microsoft.com/en-us/training/modules/describe-benefits-use-cloud-services/2-high-availability-scalability-cloud}, “Khi bạn deploy một ứng dụng, dịch vụ, hay bất kỳ tài nguyên IT nào, việc những tài nguyên đó sẵn sàng khi bạn cần là điều quan trọng. High availability tập trung vào việc đảm bảo tối đa tính sẵn sàng của hệ thống, bất kể sự gián đoạn hay sự kiện nào có thể xảy ra.” \\[0.5cm]
\noindent Khi xây dựng giải pháp của mình, ta sẽ cần tính đến các đảm bảo về tính khả dụng của dịch vụ. Azure là môi trường đám mây có tính sẵn sàng cao với sự đảm bảo về thời gian hoạt động tùy thuộc vào dịch vụ. Những đảm bảo này là một phần của thỏa thuận cấp độ dịch vụ (SLA).

\subsubsection{Scalability - Tính mở rộng}
\noindent Theo định nghĩa của Microsoft\footnote{Như trên}, “Khả năng mở rộng (scalability) đề cập đến khả năng điều chỉnh các tài nguyên để đáp ứng nhu cầu. Nếu bạn đột nhiên gặp phải lưu lượng truy cập cao và hệ thống của bạn bị quá tải thì khả năng mở rộng quy mô có nghĩa là bạn có thể bổ sung thêm tài nguyên để xử lý tốt hơn lượng tải đang gia tăng.” \\[0.5cm]
\noindent Lợi ích khác của tính mở rộng là bạn không phải trả quá nhiều tiền cho các dịch vụ. Vì đám mây là mô hình dựa trên mức tiêu dùng nên bạn chỉ trả tiền cho những gì bạn sử dụng. Nếu nhu cầu giảm, bạn có thể giảm tài nguyên và từ đó giảm chi phí. \\[0.5cm]
\noindent Scalability thường có hai loại: dọc và ngang. Mỏ rộng theo chiều dọc tập trung vào việc tăng hoặc giảm khả năng của tài nguyên. Mở rộng theo chiều ngang là thêm hoặc bớt số lượng tài nguyên.
\subsection{Phân tích các kiểu kiến trúc hệ thống}
\subsubsection{Kiến trúc monolith}
\noindent \textbf{Định nghĩa} \\[0.3cm]
\noindent Kiến trúc monolith là kiến trúc trong đó tất cả các thành phần của một ứng dụng được đặt trong một đơn vị duy nhất. Đơn vị này thường bị hạn chế trong một phiên bản thời gian chạy duy nhất của ứng dụng. Các ứng dụng truyền thống thường bao gồm giao diện web, lớp dịch vụ và lớp dữ liệu. Trong kiến trúc nguyên khối, các lớp này được kết hợp trên một phiên bản của ứng dụng.\footnote{Website: https://learn.microsoft.com/en-us/training/modules/microservices-architecture/}
