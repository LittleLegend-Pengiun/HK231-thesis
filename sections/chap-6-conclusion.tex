\chapter{Tổng kết}
\section{Kết quả đạt được}
\subsection{Đối với việc tìm hiểu và phân tích nghiệp vụ}
\noindent Căn cứ vào mục tiêu, nhiệm vụ của đề tài đã đề ra, nhóm đã thực hiện được những điều sau:
\begin{itemize}
    \item Tiến hành phân tích các yêu cầu cần có của hệ thống, xác định trọng tâm của đồ án là thực hiện hệ thống thỏa mãn được tính sẵn sàng và tính mở rộng cao.
    \item Xác định các hướng giải quyết bài toán có thể thực hiện được.
    \item Phân tích các yêu cầu chức năng và phi chức năng của hệ thống.
\end{itemize}

\subsection{Đối với cợ sở lý thuyết và công nghệ}
\noindent Dựa vào mục tiêu, yêu cầu nghiệp vụ của đồ án, nhóm đã làm được những việc sau:
\begin{itemize}
    \item Tiến hành so sánh, tham khảo, phân tích các bài viết, ví dụ của các hệ thống tương tự để đưa ra giải pháp phù hợp với điều kiện thực tế.
    \item Tìm hiểu các khái niệm, ý tưởng, kiến trúc, cách hoạt động của hệ thống Kubernetes.
    \item Tìm hiểu được cách các hệ thống microservice được đưa lên Kubernetes.
    \item Chọn ngôn ngữ lập trình, framework, công nghệ phù hợp để xây dụng hệ thống: Frontend (web), các microservice backend và database, công cụ để xây dựng hệ thống Kubernetes ở môi trường local.
\end{itemize}
\subsection{Đối với phân tích và thiết kế hệ thống}
\noindent Dựa vào các thông tin cơ sỏ lý thuyết, công nghệ đã tìm hiểu, cộng thêm các trải nghiệm thực tế thì nhóm đã đưa ra được thiết kế kiến trúc hoàn chỉnh của hệ thống:
\begin{itemize}
    \item Đưa ra được kiến trúc hoàn chỉnh cho hệ thống, tận dụng các dịch vụ, giải pháp của nền tảng Kubernetes để hoàn thành mục tiêu, nhiệm vụ đề ra trong đồ án.
    \item Tối ưu lại giải pháp cho phù hợp với điều kiện kinh tế hiện tại của mỗi cá nhân.
\end{itemize}
\subsection{Đối với quá trình phát triển ứng dụng}
\noindent Trong suốt giai đoạn Đồ án chuyên ngành, nhóm đã làm việc một cách bài bản, có quy trình, kế hoạch cụ thể và chi tiết.
\begin{itemize}
    \item Nhóm đã thực hiện đầy đủ các giai đoạn: Phân tích, tìm hiểu yêu cầu nghiệp vụ, Thiết kế hệ thống, Triển khai ở quy mô nhỏ.
    \item Công cụ tổ chức và quản lý mã nguồn: Nhóm sử dụng Git và Github để làm tăng hiệu quả công việc.
    \item Về luồng công việc: Nhóm có áp dụng scrum-agile vào việc quản lý, phân chia, theo dõi công việc của các thành viên trong nhóm, tuy nhiên cũng có điều chỉnh cho phù hợp với đặc thù của từng thành viên.
\end{itemize}
\section{Đánh giá giải pháp và kết quả đạt được}
\section{Hướng phát triển đề tài trong tương lai}