\chapter{Giới thiệu}
% \section{Ví dụ}

% \noindent Đây là một ví dụ, sau này sẽ được update nội dung sau.

% If this chapter/section has a star, it won't be in the table of content.
% \section*{Đây là section k dc thêm vào mục lục}
\section{Giới thiệu đề tài}
\noindent Trong thời đại công nghệ số và đại dịch, việc mua sắm trực tuyến đang trở nên phổ biến hơn bao giờ hết. Điều này đặt ra một thách thức lớn cho các doanh nghiệp, đòi hỏi họ phải sở hữu kênh bán hàng trực tuyến để đáp ứng nhu cầu của khách hàng. Đề tài này nhằm xây dựng một hệ thống thương mại điện tử linh hoạt, ổn định và có khả năng mở rộng trên nhiều nền tảng khác nhau. Hệ thống này cũng phải dễ dàng bảo trì và nâng cấp trong tương lai. Do đó, việc một doanh nghiệp sở hữu kênh bán hàng trên nền tảng số là vô cùng cần thiết. Mục tiêu của đề tài là xây dựng một hệ thống thương mại điện tử có tính sẵn sàng, tính mở rộng cao, có thể được đưa lên nhiều nền tảng khác nhau một cách dễ dàng, dễ bảo trì, nâng cấp trong tương lai.

\section{Mục tiêu và phạm vi của đề tài}
\noindent Hệ thống bao gồm những chức năng chính cho một trang web thương mại điện tử bán sản phẩm công nghệ và một số tính năng, đặc điểm nổi bật như tính sẵn sàng cao, tính mở rộng cao, đáp ứng được lưu lượng truy cập biến động của người dùng.

\section{Cấu trúc đồ án}
\noindent Nội dung của đồ án được trình bày bao gồm những chương sau
% {\renewcommand\labelitemi{}
% \begin{itemize}
%     \item Chương I Giới thiệu chung về đề tài
%     \item Chương II Cơ sở lý thuyết
%     \item Chương III Tổng hợp và phân tích bài toán
%     \item Chương IV Thiết kế hệ thống
%     \item Chương V Xây dựng phiên bản thử nghiệm
% \end{itemize}
% }

\begin{enumerate}[label=\Roman*., itemsep=0pt, start=1]
    \item Giới thiệu chung về đề tài
    \item Cơ sở lý thuyết
    \item Tổng hợp và phân tích bài toán
    \item Thiết kế hệ thống
    \item Xây dựng phiên bản thử nghiệm
\end{enumerate}